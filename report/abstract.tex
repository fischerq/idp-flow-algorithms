
\cleardoublepage
\addcontentsline{toc}{chapter}{Abstract / Zusammenfassung}
\selectlanguage{english}
\section*{Abstract}

This interdisciplinary project encompasses the development of interactive applications to visualise two flow algorithms: The Ford-Fulkerson algorithm for solving the maximum-flow problem, and the cycle canceling algorithm for solving the minimum-cost flow problem. This report presents the algorithms, describes the layout of the applications, and explains the technical realisation.

The aim of the applications is to facilitate the comprehension of the algorithms by allowing the step-by-step execution on any desired instance of the problem, displaying the computational phases transparently, and graphically visualising the data structures.

The implementation is done as an application for web browsers by using an existing framework which uses modern web standards like HTML5, SVG, CSS, Javascript and D3.js. In particular, multiple important components of the implementation and their interactions are described in order to give a clear overview of the code base. 


\selectlanguage{ngerman}
\section*{Zusammenfassung}

Dieses interdisziplinäre Projekt umfasst die Entwicklung von interaktiven Anwendungen für die Visualisierung zweier Flussalgorithmen: Des Ford-Fulkerson Algorithmus zur Lösung des Maximum-Flow Problems, und des Cycle-Canceling Algorithmus zur Lösung des Minimum-Cost Flow Problems. Dieser Bericht stellt die Algorithmen vor, beschreibt den Aufbau der Applikationen, und erläutert die technische Umsetzung. 

Das Ziel der Applikationen ist das Verständnis der Algorithmen zu erleichtern, indem eine schrittweise Ausführung an beliebigen Instanzen des Problems ermöglicht, die Berechnungsschritte sichtbar gemacht und Datenstrukturen anschaulich visualisiert werden.

Die Implementierung erfolgt als Webapplikation für den Browser mithilfe eines bestehenden Frameworks, das moderne Webstandards wie HTML5, SVG, CSS, Javascript und D3.js verwendet. Insbesondere werden verschiedene wichtigen Komponenten der Implementierung und deren Zusammenspiel behandelt, um einen klaren Überblick über den Code zu geben.

\selectlanguage{ngerman}

