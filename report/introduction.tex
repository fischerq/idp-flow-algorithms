%% ==============================
\addcontentsline{toc}{chapter}{Einleitung}
\chapter*{Einleitung}
%% ==============================

Die Graphentheorie ist eine Teilgebiet der (diskreten) Mathematik mit einigen herausstellenden Merkmalen: Sie ist sehr anwendungsbezogen, da viele praktische Entscheidungsprobleme mithilfe von Graphen modelliert werde können. Sie hat weiterhin eine sehr enge Verbindung zur Algorithmik, wobei sowohl algorithmische Probleme mit bestimmten Graphen zusammenhängen, als auch Probleme der Graphentheorie mithilfe von Algorithmen gelöst werden, und daraus folgend auch zur Informatik. Aufgrund ihrer Anschaulichkeit und der Nützlichkeit der Konzepte ist die Graphentheorie einer der ersten Bereiche der Mathematik mit dem Studenten mathematischer, informatischer oder technischer Fachrichtungen in Berührung kommen.

Eine besonders charakteristischer Teil hierbei ist die Beschäftigung mit verschiedenen Graphalgorithmen. Das Verständnis der Abläufe, der Eigenschaften und Strukturen die der einzelne Algorithmus ausnutzt, und den entsprechenden Eigenschaften der resultierenden Lösungen gibt eine vielseitige Perspektive auf die behandelten Problemstellungen. Eine wichtiges Hilfsmittel für die Einarbeitung in neue Algorithmen sind dabei die ausführliche Durchführung des Algorithmus auf verschiedenen Beispielen und Betrachtung der Berechnungsschritte. Insbesondere die Visualisierung des Graphen mit verschiedenen Annotationen ist vielmals hilfreich um ein Gefühl für die Vorgänge zu bekommen.

Eine Klasse von Graphalgorithmen mit vielseitigen Anwendungen sind Flussprobleme. Abgeleitet von praktischen Fragestellungen zum Transport in verschiedenartigen Netzwerken, man denke an Rohrnetze, Straßennetze oder Datennetzwerke, behandeln sie Optimierungsprobleme wie eine Menge an Ressourcen von einem Start- zu einem Zielknoten gelangen kann. Je nach speziellen Anforderungen wird der Graph, der die Netzwerktopologie modelliert, mit verschiedenen Informationen annotiert. Typischerweise sind dies Kapazitäten (verschiedene Teile des Netzwerkes können nur bestimmte Mengen an Ressource transportieren) oder Kosten (die Benutzung mancher Teile des Netzwerks is anderen vorzuziehen).

Dieses Dokument beschreibt die Entwicklung einer interaktiven Anwendung zur Visualisierung zweier Flussalgorithmen - des Ford-Fulkerson Algorithmus für das Maximum-Flow Problem, und des Cycle Canceling Algorithmus für das Min-Cost Flow Problem. Die Anwendung hat das Ziel das Verständnis der Algorithmen zu erleichtern, indem sie einfach Schritt für Schritt auf verschiedene Instanzen des Problems ausgeführt werden können, mit transparenter Anzeige des Berechnungszustandes und anschaulicher Darstellung der Datenstrukturen. 

Im erste Kapitel werden die zwei behandelten Probleme und Lösungsalgorithmen behandelt. Im zweiten Kapitel wird der Aufbau der Anwendung und ihre verschiedenen Elemente beschrieben. Das dritte Kapitel beschreibt die technische Umsetzung, mit zunächst einem Überblick über die verwendeten Basistechnologien und schließlich einer Behandlung der Implementierung interessanter Komponenten und deren Zusammenspiel.